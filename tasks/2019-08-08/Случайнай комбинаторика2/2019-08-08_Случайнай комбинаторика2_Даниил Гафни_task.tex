\documentclass[a4paper,11pt]{article}%
\usepackage[T1]{fontenc}%
\usepackage[utf8]{inputenc}%
\usepackage{lmodern}%
\usepackage{textcomp}%
\usepackage{lastpage}%
\usepackage[english,russian]{babel}%
\usepackage{mathtext}%
\usepackage[a4paper,margin=1.5truecm,top=1.3truecm,bottom=1.0truecm]{geometry}%
\usepackage[T2A]{fontenc}%
\usepackage{amsmath}%
\usepackage{amsthm}%
\usepackage{amsfonts}%
\usepackage{mathabx}%
\usepackage{graphicx}%
\usepackage{tabularx}%
%
\theoremstyle{definition}%
\newtheorem{Ex}{\hspace{-25pt}\fbox{\phantom{123}} Задача}%
\title{Случайнай комбинаторика2}%
\author{Даниил Гафни}%
\date{2019{-}08{-}08}%
%
\begin{document}%
\normalsize%
\maketitle%
\begin{Ex}%
Сколько существует способов поставить двух черных ферзей на доску?%
%
\end{Ex}%
%
\iffalse%
Автор: Нина%
Дата: 2013%
Название: %
Подсказка: \textbackslash{}\textbackslash{}%
nan%
\fi%
%
\begin{Ex}%
В ленту длиной десять клеток вписывают числа от 1 до 10. При чем каждое следующее приписывают только рядом с уже существующим числом. Сколько способов заполнить таблицу существует?%
%
\end{Ex}%
%
\iffalse%
Автор: Нина%
Дата: 2013%
Название: %
Подсказка: \textbackslash{}\textbackslash{}%
nan%
\fi%
%
\begin{Ex}%
В ленту длиной десять клеток вписывают числа от 1 до 10. При чем каждое следующее приписывают только рядом с уже существующим числом. Сколько способов заполнить таблицу существует?%
%
\end{Ex}%
%
\iffalse%
Автор: Нина%
Дата: 2013%
Название: %
Подсказка: \textbackslash{}\textbackslash{}%
nan%
\fi%
%
\begin{Ex}%
В меню «Шоколадницы» есть 4 вида блинчиков и 3 различных молочных коктейля.
Сколькими способами сможете выбрать «Блинчики + коктейль»?%
%
\end{Ex}%
%
\iffalse%
Автор: ЗМШ%
Дата: 2012%
Название: %
Подсказка: \textbackslash{}\textbackslash{}%
nan%
\fi%
%
\begin{Ex}%
В вазе лежат 5 шоколадок с разными начинками и 7 мармеладок (всех цветов
радуги). Но можно выбрать только одну сладость. Сколько способов у вас есть? А
если можно выбрать одну шоколадку И одну мармеладку?%
%
\end{Ex}%
%
\iffalse%
Автор: ЗМШ%
Дата: 2012%
Название: %
Подсказка: \textbackslash{}\textbackslash{}%
nan%
\fi%
%
\begin{Ex}%
А сколько способов есть выбрать 4 шоколадки из этих 5?%
%
\end{Ex}%
%
\iffalse%
Автор: ЗМШ%
Дата: 2012%
Название: %
Подсказка: \textbackslash{}\textbackslash{}%
nan%
\fi%
%
\begin{Ex}%
Сколькими способами можно переставить буквы в слове ПРОБЛЕСК так, чтобы буква П шла непо-
средственно после буквы О?%
%
\end{Ex}%
%
\iffalse%
Автор: ЗМШ%
Дата: 2013%
Название: %
Подсказка: \textbackslash{}\textbackslash{}%
nan%
\fi%
%
\begin{Ex}%
Ладья стоит на левом поле клетчатой полоски 1 на 30 и за ход может сдвинуться на любое количество
клеток вправо. Сколькими способами она может добраться до крайнего правого поля?%
%
\end{Ex}%
%
\iffalse%
Автор: ЗМШ%
Дата: 2013%
Название: %
Подсказка: \textbackslash{}\textbackslash{}%
nan%
\fi%
%
\begin{Ex}%
Выписаны всевозможные строки из нулей и единиц длины 8. Сколько всего выписано единиц?%
%
\end{Ex}%
%
\iffalse%
Автор: ЗМШ%
Дата: 2013%
Название: %
Подсказка: \textbackslash{}\textbackslash{}%
nan%
\fi%
%
\begin{Ex}%
Сколькими способами можно поставить на шахматную доску
а) черную и белую ладьи; б) две белых ладьи так, чтобы они не били друг друга?%
%
\end{Ex}%
%
\iffalse%
Автор: ЗМШ%
Дата: 2013%
Название: %
Подсказка: \textbackslash{}\textbackslash{}%
nan%
\fi%
%
\begin{Ex}%
Сколькими способами можно поставить на шахматную доску
а) черную и белую ладьи; б) две белых ладьи так, чтобы они не били друг друга?%
%
\end{Ex}%
%
\iffalse%
Автор: ЗМШ%
Дата: 2013%
Название: %
Подсказка: \textbackslash{}\textbackslash{}%
nan%
\fi%
%
\begin{Ex}%
а) В команде из 5 человек нужно выбрать капитана и его заместителя. Сколько есть способов это
сделать?
б) То же, но нужно выбрать двух капитанов.
в) То же, но нужно выбрать капитана, старшего помощника и младшего помощника.
г) То же, но нужно выбрать капитана и двух его помощников.
д) То же, но нужно выбрать трјх капитанов.%
%
\end{Ex}%
%
\iffalse%
Автор: ЗМШ%
Дата: 2013%
Название: %
Подсказка: \textbackslash{}\textbackslash{}%
nan%
\fi%
%
\begin{Ex}%
Какое наибольшее количество а) ладей; б) королей; в) коней; г) слонов можно поставить на шахматной доске так, чтобы никакие две фигуры друг друга не били?%
%
\end{Ex}%
%
\iffalse%
Автор: ЗМШ%
Дата: 2013%
Название: %
Подсказка: \textbackslash{}\textbackslash{}%
nan%
\fi%
%
\begin{Ex}%
Какое наибольшее количество а) ладей; б) королей; в) коней; г) слонов можно поставить на шахматной доске так, чтобы никакие две фигуры друг друга не били?%
%
\end{Ex}%
%
\iffalse%
Автор: ЗМШ%
Дата: 2013%
Название: %
Подсказка: \textbackslash{}\textbackslash{}%
nan%
\fi%
%
\begin{Ex}%
Можно ли в клетки таблички 6 на 7 вписать 20 крестиков и 22 нолика так, чтобы в каждом квадрате 2 на 2 и в каждом прямоугольнике 1 на 4 стояло хотя бы два крестика?%
%
\end{Ex}%
%
\iffalse%
Автор: ЗМШ%
Дата: 2013%
Название: %
Подсказка: \textbackslash{}\textbackslash{}%
nan%
\fi%
%
\begin{Ex}%
Из двух математиков и десяти экономистов надо составить комиссию из восьми человек. Сколькими способами можно составить комиссию, если в нее должен входить хотя бы один математик? %
%
\end{Ex}%
%
\iffalse%
Автор: Лев Юманов%
Дата: 2016%
Название: %
Подсказка: \textbackslash{}\textbackslash{}%
nan%
\fi%
%
\begin{Ex}%
Найдите сумму всех семизначных чисел, которые можно получить всевозможными перестановками цифр 1,...,7%
%
\end{Ex}%
%
\iffalse%
Автор: Лев Юманов%
Дата: 2016%
Название: %
Подсказка: \textbackslash{}\textbackslash{}%
nan%
\fi%
%
\begin{Ex}%
Сколько существует двузначных чисел, у которых цифры идут по возрастанию?%
%
\end{Ex}%
%
\iffalse%
Автор: Ираклий Гагуа%
Дата: 2011%
Название: %
Подсказка: \textbackslash{}\textbackslash{}%
nan%
\fi%
%
\begin{Ex}%
Сколько существует трехзначных чисел, у которых цифры идут по возрастанию?%
%
\end{Ex}%
%
\iffalse%
Автор: Ираклий Гагуа%
Дата: 2011%
Название: %
Подсказка: \textbackslash{}\textbackslash{}%
nan%
\fi%
%
\begin{Ex}%
Сколькими способами можно выбрать гласную и согласную буквы в слове «КРУЖОК»?%
%
\end{Ex}%
%
\iffalse%
Автор: Ираклий Гагуа%
Дата: 2011%
Название: %
Подсказка: \textbackslash{}\textbackslash{}%
nan%
\fi%
%
\begin{Ex}%
На доске написаны 7 существительных, 5 глаголов и 2 прилагательных. Для предложения нужно выбрать по одному слову каждой из этих частей речи. Сколькими способами это можно сделать?%
%
\end{Ex}%
%
\iffalse%
Автор: Ираклий Гагуа%
Дата: 2011%
Название: %
Подсказка: \textbackslash{}\textbackslash{}%
nan%
\fi%
%
\begin{Ex}%
Сколькими способами можно поставить на доску 3х3 белую и чёрную ладьи так, чтобы они не били друг друга?%
%
\end{Ex}%
%
\iffalse%
Автор: Ираклий Гагуа%
Дата: 2011%
Название: %
Подсказка: \textbackslash{}\textbackslash{}%
nan%
\fi%
%
\begin{Ex}%
Сколькими способами можно поставить на доску 3х3 двух белых ладей так, чтобы они не били друг друга?%
%
\end{Ex}%
%
\iffalse%
Автор: Ираклий Гагуа%
Дата: 2011%
Название: %
Подсказка: \textbackslash{}\textbackslash{}%
nan%
\fi%
%
\begin{Ex}%
Каких двоичных слов длины 50 больше: тех, в которых 36 единиц, или тех, в которых 14 единиц?%
%
\end{Ex}%
%
\iffalse%
Автор: Ираклий Гагуа%
Дата: 2011%
Название: %
Подсказка: \textbackslash{}\textbackslash{}%
nan%
\fi%
%
\begin{Ex}%
На дискотеку пришли 10 юношей и 10 девушек. Сколькими способами они могут разбиться на пары перед очередным медленным танцем?%
%
\end{Ex}%
%
\iffalse%
Автор: 25 школа%
Дата: 2017%
Название: %
Подсказка: \textbackslash{}\textbackslash{}%
nan%
\fi%
%
\begin{Ex}%
Сколькими способами можно 20 человек разбить на пары?%
%
\end{Ex}%
%
\iffalse%
Автор: 25 школа%
Дата: 2017%
Название: %
Подсказка: \textbackslash{}\textbackslash{}%
nan%
\fi%
%
\begin{Ex}%
Сколькими способами можно переставить n подносов, стоящих на длинном узком столе? Это называется число перестановок из n элементов и обозначается $P_n$.%
%
\end{Ex}%
%
\iffalse%
Автор: Ираклий Гагуа%
Дата: 2011%
Название: %
Подсказка: \textbackslash{}\textbackslash{}%
nan%
\fi%
%
\begin{Ex}%
Сколькими способами можно переставить цифры в числе 75571193 так, чтобы в полученном
числе обязательно присутствовало сочетание 57?%
%
\end{Ex}%
%
\iffalse%
Автор: Павел Живцов%
Дата: 2018%
Название: %
Подсказка: \textbackslash{}\textbackslash{}%
nan%
\fi%
%
\begin{Ex}%
Сколькими способами можно из 15 слов выбрать набор, состоящий не более чем
из 5 слов?%
%
\end{Ex}%
%
\iffalse%
Автор: 25 школа%
Дата: 2017%
Название: %
Подсказка: \textbackslash{}\textbackslash{}%
nan%
\fi%
%
\begin{Ex}%
Семен имеет 6 друзей и в течение 5 дней приглашает в гости каких–то троих
из них так, чтобы компания ни разу не повторялась. Сколькими способами он может
это сделать?%
%
\end{Ex}%
%
\iffalse%
Автор: 25 школа%
Дата: 2017%
Название: %
Подсказка: \textbackslash{}\textbackslash{}%
nan%
\fi%
%
\begin{Ex}%
Сколько существует десятизначных чисел, у которых сумма цифр равна 4?%
%
\end{Ex}%
%
\iffalse%
Автор: 25 школа%
Дата: 2017%
Название: %
Подсказка: \textbackslash{}\textbackslash{}%
nan%
\fi%
%
\begin{Ex}%
В классе 10 учеников. Сколькими способами можно разбить этот класс а)
на две; б) на три не обязательно равные непустые группы? (Назначения
групп различны!)%
%
\end{Ex}%
%
\iffalse%
Автор: 25 школа%
Дата: 2012%
Название: %
Подсказка: \textbackslash{}\textbackslash{}%
nan%
\fi%
%
\begin{Ex}%
Сколько решений в натуральных числах имеет уравнение $x y z=10^{9}$?%
%
\end{Ex}%
%
\iffalse%
Автор: Дмитрий Трущин%
Дата: 2013%
Название: %
Подсказка: \textbackslash{}\textbackslash{}%
nan%
\fi%
%
\begin{Ex}%
6 карточек пронумерованы числами 1, 2, 3, 4, 5, 6.
Карточки наугад выкладываем в ряд. Сколько при этом можно получить
различных шестизначных чисел?%
%
\end{Ex}%
%
\iffalse%
Автор: Дарья Анзон%
Дата: 2017%
Название: %
Подсказка: \textbackslash{}\textbackslash{}%
nan%
\fi%
%
\begin{Ex}%
Посмотрите на строчки треугольника Паскаля.
(a) Выпишите явно, чему равно $(x + 1)^4$ и $(x + 1)^5$.
(b) Докажите, используя только реккурентное определение треугольника Паскаля, связь между биномом Ньютона и треугольником Паскаля.%
%
\end{Ex}%
%
\iffalse%
Автор: Павел Минаев%
Дата: 2017%
Название: %
Подсказка: \textbackslash{}\textbackslash{}%
nan%
\fi%
%
\begin{Ex}%
Сколько различных чисел можно составить из цифр 0, 1, 2, 3, если ни одна из
цифр не будет повторяться? (число с нуля начинаться не может)%
%
\end{Ex}%
%
\iffalse%
Автор: Дарья Анзон%
Дата: 2017%
Название: %
Подсказка: \textbackslash{}\textbackslash{}%
nan%
\fi%
%
\begin{Ex}%
Решите уравнения: а) $\frac{(n+1) !}{(n-1) !}=72$ б)$\frac{n !-(n-1) !}{(n+1) !}=\frac{1}{6}$%
%
\end{Ex}%
%
\iffalse%
Автор: Дарья Анзон%
Дата: 2017%
Название: %
Подсказка: \textbackslash{}\textbackslash{}%
nan%
\fi%
%
\begin{Ex}%
Семь девушек водят хоровод. Сколькими различными способами они могут
встать в круг?%
%
\end{Ex}%
%
\iffalse%
Автор: Дарья Анзон%
Дата: 2017%
Название: %
Подсказка: \textbackslash{}\textbackslash{}%
nan%
\fi%
%
\begin{Ex}%
Человек забыл две последние цифры в семизначном телефонном номере,
помнит только, что все цифры номера были неодинаковые. Сколько
телефонных номеров ему придется опробовать, чтобы дозвониться?%
%
\end{Ex}%
%
\iffalse%
Автор: Дарья Анзон%
Дата: 2017%
Название: %
Подсказка: \textbackslash{}\textbackslash{}%
nan%
\fi%
%
\begin{Ex}%
В этой задаче нужно внимательно смотреть на треугольник Паскаля
(a) Пусть есть бесконечная шахматная доска вправо и вверх. В левом нижнем углу стоит хромой король,
умеющий ходить только вправо и вверх. Сколько есть способов добраться до клетки (n;m), где n – число
строк по вертикали, а m – по горизонтали?
(b) Напишем в каждую клетку доски из предыдущей задачи число способов добраться до этой клетки. Как
связан треугольник Паскаля и эта табличка с числами? Докажите свою гипотезу.%
%
\end{Ex}%
%
\iffalse%
Автор: Павел Минаев%
Дата: 2017%
Название: %
Подсказка: \textbackslash{}\textbackslash{}%
nan%
\fi%
%
\end{document}