\documentclass{article}%
\usepackage[T1]{fontenc}%
\usepackage[utf8]{inputenc}%
\usepackage{lmodern}%
\usepackage{textcomp}%
\usepackage{lastpage}%
\usepackage[tmargin=1cm,lmargin=4cm]{geometry}%
\usepackage[english,russian]{babel}%
\usepackage[dvipsnames]{xcolor}%
%
\title{Ответы к линейное возрастание сложности}%
\author{Даниил Гафни}%
\date{2019{-}07{-}26}%
%
\begin{document}%
\normalsize%
\maketitle%
\section{(сложность {-} 1)}%
\label{sec:( {-} 1)}%
\hspace{3ex} Ответ: 6.28 \\%
%
\hspace*{3ex} Решение: \\%
Посчитаем длину окружности: \\ $2 \pi 1 = 6.28$

%
\section{(сложность {-} 2)}%
\label{sec:( {-} 2)}%
\hspace{3ex} Ответ: 12 \\%
%
\hspace*{3ex} Решение: \\%
Периметр квадрата равен $4a = 4 \dot 3 = 12$ 

%
\end{document}