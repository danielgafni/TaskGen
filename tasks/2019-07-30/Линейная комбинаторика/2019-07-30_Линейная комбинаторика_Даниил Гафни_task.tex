\documentclass{article}%
\usepackage[T1]{fontenc}%
\usepackage[utf8]{inputenc}%
\usepackage{lmodern}%
\usepackage{textcomp}%
\usepackage{lastpage}%
\usepackage[tmargin=1cm,lmargin=4cm]{geometry}%
\usepackage[english,russian]{babel}%
\usepackage[dvipsnames]{xcolor}%
%
\title{Линейная комбинаторика}%
\author{Даниил Гафни}%
\date{2019{-}07{-}30}%
%
\begin{document}%
\normalsize%
\maketitle%
\section{}%
\label{sec:}%
Сколько существует
А) трехзначных чисел?
Б) пятизначных чисел?
В) пятизначных чисел, в которых все цифры четные?
Г) пятизначных чисел, в которых есть хотя бы одна четная цифра?%
%
\iffalse%
Автор: Ираклий Гагуа%
Дата: 07{-}01{-}2011%
Название: None%
Подсказка: \textbackslash{}\textbackslash{}%
nan%
\fi

%
\section{}%
\label{sec:}%
Сколькими способами можно 20 человек разбить на пары?%
%
\iffalse%
Автор: 25 школа%
Дата: 01{-}07{-}2017%
Название: None%
Подсказка: \textbackslash{}\textbackslash{}%
nan%
\fi

%
\section{}%
\label{sec:}%
В классе из 25 человек требуется выбрать а) старосту и его заместителя; б)
двух дежурных. Сколькими способами это можно сделать?%
%
\iffalse%
Автор: 25 школа%
Дата: 01{-}11{-}2012%
Название: None%
Подсказка: \textbackslash{}\textbackslash{}%
nan%
\fi

%
\section{}%
\label{sec:}%
В классе 10 учеников. Сколькими способами можно разбить этот класс а)
на две; б) на три не обязательно равные непустые группы? (Назначения
групп различны!)%
%
\iffalse%
Автор: 25 школа%
Дата: 01{-}11{-}2012%
Название: None%
Подсказка: \textbackslash{}\textbackslash{}%
nan%
\fi

%
\section{}%
\label{sec:}%
Сколько существует трехзначных чисел, у которых цифры идут по возрастанию?%
%
\iffalse%
Автор: Ираклий Гагуа%
Дата: 07{-}01{-}2011%
Название: None%
Подсказка: \textbackslash{}\textbackslash{}%
nan%
\fi

%
\section{}%
\label{sec:}%
В алфавите племени Умбо–Юмбо всего три буквы: А, У и О. Словом
считается любая последовательность букв, содержащая не более, чем 4
буквы. Сколько слов в языке племени Умбо–Юмбо?%
%
\iffalse%
Автор: 25 школа%
Дата: 01{-}11{-}2012%
Название: None%
Подсказка: \textbackslash{}\textbackslash{}%
nan%
\fi

%
\section{}%
\label{sec:}%
Назовем число симпатичным, если в его записи встречаются только
нечетные цифры. Сколько существует а) пятизначных симпатичных чисел;
б) пятизначниых несимпатичных чисел?%
%
\iffalse%
Автор: 25 школа%
Дата: 01{-}11{-}2012%
Название: None%
Подсказка: \textbackslash{}\textbackslash{}%
nan%
\fi

%
\section{}%
\label{sec:}%
Монету бросают трижды. Сколько различных последовательностей орлов
и решек можно получить? А если монету бросать 100 раз?%
%
\iffalse%
Автор: 25 школа%
Дата: 01{-}11{-}2012%
Название: None%
Подсказка: \textbackslash{}\textbackslash{}%
nan%
\fi

%
\section{}%
\label{sec:}%
Докажите, что количество способов выбрать из n элементов  группу из k равно n!/(k!(n-k)!)%
%
\iffalse%
Автор: 25 школа%
Дата: 01{-}07{-}2017%
Название: None%
Подсказка: \textbackslash{}\textbackslash{}%
nan%
\fi

%
\section{}%
\label{sec:}%
Кощей Бессмертный, желая сделать Бабе Яге подарок на Новый Год,
приобрел кучу метелок трех сортов, ступы 5 видов и головные платки 7
расцветок. Он хочет каждый Новый Год дарить Яге 1 метлу, 1 ступу и 1
платок, но так, чтобы ни один год наборы подарков не совпадали. На
сколько лет ему хватит приобретенных товаров? (Считайте, что
количество приобретенных предметов сколь угодно велико.)%
%
\iffalse%
Автор: 25 школа%
Дата: 01{-}11{-}2012%
Название: None%
Подсказка: \textbackslash{}\textbackslash{}%
nan%
\fi

%
\section{}%
\label{sec:}%
В магазин «Всё для чая» завезли 4 различные чайные ложки.
а) Сколькими способами можно купить набор из трех предметов?
б) А из двух предметов?%
%
\iffalse%
Автор: Ираклий Гагуа%
Дата: 07{-}01{-}2011%
Название: None%
Подсказка: \textbackslash{}\textbackslash{}%
nan%
\fi

%
\section{}%
\label{sec:}%
Сколько существует десятизначных чисел, у которых сумма цифр равна 4?%
%
\iffalse%
Автор: 25 школа%
Дата: 01{-}07{-}2017%
Название: None%
Подсказка: \textbackslash{}\textbackslash{}%
nan%
\fi

%
\section{}%
\label{sec:}%
На доске написаны 7 существительных, 5 глаголов и 2 прилагательных. Для предложения нужно выбрать по одному слову каждой из этих частей речи. Сколькими способами это можно сделать?%
%
\iffalse%
Автор: Ираклий Гагуа%
Дата: 07{-}01{-}2011%
Название: None%
Подсказка: \textbackslash{}\textbackslash{}%
nan%
\fi

%
\section{}%
\label{sec:}%
В команде из 11 человек нужно выбрать капитана и заместителя. Сколькими способами это можно сделать?%
%
\iffalse%
Автор: Ираклий Гагуа%
Дата: 07{-}01{-}2011%
Название: None%
Подсказка: \textbackslash{}\textbackslash{}%
nan%
\fi

%
\section{}%
\label{sec:}%
Сколько существует двоичных слов длины n, в которых ровно k нулей? Это называется число сочетаний из n элементов по k и обозначается $C_n^k$ или $(^n_k)$%
%
\iffalse%
Автор: Ираклий Гагуа%
Дата: 07{-}01{-}2011%
Название: None%
Подсказка: \textbackslash{}\textbackslash{}%
nan%
\fi

%
\section{}%
\label{sec:}%
Сколько существует двузначных чисел, у которых цифры идут по возрастанию?%
%
\iffalse%
Автор: Ираклий Гагуа%
Дата: 07{-}01{-}2011%
Название: None%
Подсказка: \textbackslash{}\textbackslash{}%
nan%
\fi

%
\section{}%
\label{sec:}%
Каких двоичных слов длины 50 больше: тех, в которых 36 единиц, или тех, в которых 14 единиц?%
%
\iffalse%
Автор: Ираклий Гагуа%
Дата: 07{-}01{-}2011%
Название: None%
Подсказка: \textbackslash{}\textbackslash{}%
nan%
\fi

%
\section{}%
\label{sec:}%
Сколько диагоналей имеет выпуклый а) 25-тиугольник б) n-угольник?%
%
\iffalse%
Автор: 25 школа%
Дата: 01{-}11{-}2012%
Название: None%
Подсказка: \textbackslash{}\textbackslash{}%
nan%
\fi

%
\section{}%
\label{sec:}%
Сколько существует двоичных слов длины 10, в которых ровно 4 нуля?%
%
\iffalse%
Автор: Ираклий Гагуа%
Дата: 07{-}01{-}2011%
Название: None%
Подсказка: \textbackslash{}\textbackslash{}%
nan%
\fi

%
\section{}%
\label{sec:}%
На дискотеку пришли 10 юношей и 10 девушек. Сколькими способами они могут разбиться на пары перед очередным медленным танцем?%
%
\iffalse%
Автор: 25 школа%
Дата: 01{-}07{-}2017%
Название: None%
Подсказка: \textbackslash{}\textbackslash{}%
nan%
\fi

%
\section{}%
\label{sec:}%
Сколькими способами можно поставить на доску 3х3 белую и чёрную ладьи так, чтобы они не били друг друга?%
%
\iffalse%
Автор: Ираклий Гагуа%
Дата: 07{-}01{-}2011%
Название: None%
Подсказка: \textbackslash{}\textbackslash{}%
nan%
\fi

%
\section{}%
\label{sec:}%
В 7М классе учится 26 человек. Сколькими способами их можно
выстроить в ряд при условии, что а) Ваня и Женя должны обязательно
стоять рядом; б) Ваня и Женя ни в коем случае не должны стоять рядом.%
%
\iffalse%
Автор: 25 школа%
Дата: 01{-}11{-}2012%
Название: None%
Подсказка: \textbackslash{}\textbackslash{}%
nan%
\fi

%
\section{}%
\label{sec:}%
Каких семизначных чисел больше: тех, в записи которых есть 1 или
остальных?%
%
\iffalse%
Автор: 25 школа%
Дата: 01{-}11{-}2012%
Название: None%
Подсказка: \textbackslash{}\textbackslash{}%
nan%
\fi

%
\section{}%
\label{sec:}%
Сколькими способами можно выбрать гласную и согласную буквы в слове «КРУЖОК»?%
%
\iffalse%
Автор: Ираклий Гагуа%
Дата: 07{-}01{-}2011%
Название: None%
Подсказка: \textbackslash{}\textbackslash{}%
nan%
\fi

%
\section{}%
\label{sec:}%
Сколькими способами можно поставить на шахматную доску а) белого и
черного королей; б) двух белых королей так, чтобы они не били друг
друга?%
%
\iffalse%
Автор: 25 школа%
Дата: 01{-}11{-}2012%
Название: None%
Подсказка: \textbackslash{}\textbackslash{}%
nan%
\fi

%
\section{}%
\label{sec:}%
Рота состоит из 3 лейтенантов, 6 сержантов и 60 рядовых. Сколькими способами
можно выбрать из них отряд, состоящий из одного лейтенанта, двух сержантов и 20
рядовых?%
%
\iffalse%
Автор: 25 школа%
Дата: 01{-}07{-}2017%
Название: None%
Подсказка: \textbackslash{}\textbackslash{}%
nan%
\fi

%
\section{}%
\label{sec:}%
Сколько различных слов (не обязательно осмысленных) можно составить, используя 3 буквы А, 4 буквы Б и 5 букв В?%
%
\iffalse%
Автор: Ираклий Гагуа%
Дата: 07{-}01{-}2011%
Название: None%
Подсказка: \textbackslash{}\textbackslash{}%
nan%
\fi

%
\section{}%
\label{sec:}%
Сколькими способами можно из 15 слов выбрать набор, состоящий не более чем
из 5 слов?%
%
\iffalse%
Автор: 25 школа%
Дата: 01{-}07{-}2017%
Название: None%
Подсказка: \textbackslash{}\textbackslash{}%
nan%
\fi

%
\section{}%
\label{sec:}%
В группе Платоны учатся 2 девочки и 8 мальчиков. Сколькими способами
Дмитрий Владимирович может отправить на олимпиаду команду из четырех человек
так, чтобы среди участников была хотя бы одна девочка?%
%
\iffalse%
Автор: 25 школа%
Дата: 01{-}07{-}2017%
Название: None%
Подсказка: \textbackslash{}\textbackslash{}%
nan%
\fi

%
\section{}%
\label{sec:}%
Сколькими способами можно 10 человек разбить на 2 баскетбольные команды
по 5 в каждой?%
%
\iffalse%
Автор: 25 школа%
Дата: 01{-}07{-}2017%
Название: None%
Подсказка: \textbackslash{}\textbackslash{}%
nan%
\fi

%
\section{}%
\label{sec:}%
Сколькими способами можно расставить 12 белых и 12 черных шашек на черных
полях шахматной доски?%
%
\iffalse%
Автор: 25 школа%
Дата: 01{-}07{-}2017%
Название: None%
Подсказка: \textbackslash{}\textbackslash{}%
nan%
\fi

%
\section{}%
\label{sec:}%
Имеется $k_1$ шариков 1-го цвета, … , $k_n$ шариков n-го цвета. Сколькими способами их можно расставить в ряд?%
%
\iffalse%
Автор: Ираклий Гагуа%
Дата: 07{-}01{-}2011%
Название: None%
Подсказка: \textbackslash{}\textbackslash{}%
nan%
\fi

%
\section{}%
\label{sec:}%
Сколькими способами можно поставить на доску 3х3 двух белых ладей так, чтобы они не били друг друга?%
%
\iffalse%
Автор: Ираклий Гагуа%
Дата: 07{-}01{-}2011%
Название: None%
Подсказка: \textbackslash{}\textbackslash{}%
nan%
\fi

%
\section{}%
\label{sec:}%
Сколько существует шестизначных чисел у которых четных и нечетных цифр
поровну?%
%
\iffalse%
Автор: 25 школа%
Дата: 01{-}07{-}2017%
Название: None%
Подсказка: \textbackslash{}\textbackslash{}%
nan%
\fi

%
\section{}%
\label{sec:}%
Семен имеет 6 друзей и в течение 5 дней приглашает в гости каких–то троих
из них так, чтобы компания ни разу не повторялась. Сколькими способами он может
это сделать?%
%
\iffalse%
Автор: 25 школа%
Дата: 01{-}07{-}2017%
Название: None%
Подсказка: \textbackslash{}\textbackslash{}%
nan%
\fi

%
\section{}%
\label{sec:}%
Сколькими способами можно переставить 7 подносов, стоящих на длинном узком столе?%
%
\iffalse%
Автор: Ираклий Гагуа%
Дата: 07{-}01{-}2011%
Название: None%
Подсказка: \textbackslash{}\textbackslash{}%
nan%
\fi

%
\section{}%
\label{sec:}%
Имеется 6 разноцветных коробок. Сколькими способами можно в эти коробки
разложить 20 шаров при условии, что:
а. Шары разных размеров.
б. Шары одинаковые.
в. Шары одинаковые, и не должно остаться пустых коробок.
г. Шары одинаковые, и в каждой коробке должно оказаться не менее двух шаров.%
%
\iffalse%
Автор: Дмитрий Трущин%
Дата: 01{-}10{-}2013%
Название: None%
Подсказка: \textbackslash{}\textbackslash{}%
nan%
\fi

%
\section{}%
\label{sec:}%
На прямой отмечено 10 точек, а на параллельной ей прямой –11. Сколько
существует трапеций с вершинами в этих точках?%
%
\iffalse%
Автор: 25 школа%
Дата: 01{-}07{-}2017%
Название: None%
Подсказка: \textbackslash{}\textbackslash{}%
nan%
\fi

%
\section{}%
\label{sec:}%
Сколько решений в целых числах имеет уравнение a + b + c + d + e + f = 20 при
условии, что
а) $a, b, c, d, e, f \geq 0$ б)$a, b, c, d, e, f \geq 1$ в) $a, b, c, d, e, f \geq 2$%
%
\iffalse%
Автор: Дмитрий Трущин%
Дата: 01{-}10{-}2013%
Название: None%
Подсказка: \textbackslash{}\textbackslash{}%
nan%
\fi

%
\section{}%
\label{sec:}%
Сколькими способами натуральное число n можно представить ввиде суммы k
а) неотрицательных; б) положительных слагаемых?%
%
\iffalse%
Автор: Дмитрий Трущин%
Дата: 01{-}10{-}2013%
Название: None%
Подсказка: \textbackslash{}\textbackslash{}%
nan%
\fi

%
\section{}%
\label{sec:}%
Переплетчик должен переплести 12 одинаковых книг в красный, синий или
зеленый переплеты. Сколькими способами он может это сделать?%
%
\iffalse%
Автор: Дмитрий Трущин%
Дата: 01{-}10{-}2013%
Название: None%
Подсказка: \textbackslash{}\textbackslash{}%
nan%
\fi

%
\section{}%
\label{sec:}%
В почтовом отделении продаются открыти 10 видов. Сколькими способами
можно купить а) 8 различных открыток; б) 12 открыток?%
%
\iffalse%
Автор: Дмитрий Трущин%
Дата: 01{-}10{-}2013%
Название: None%
Подсказка: \textbackslash{}\textbackslash{}%
nan%
\fi

%
\section{}%
\label{sec:}%
В стране живет n человек. Президента выбирают посредством тайного
голосования, при котором каждый гражданин голосует за одного из граждан
(возможно, за себя самого). Сколькими способами может быть составлен протокол
этого голосования? (Тайным называется голосование, при котором известно, сколько
голосов подано за каждого из кандидатов, но не известно, кто именно как
проголосовал.)%
%
\iffalse%
Автор: Дмитрий Трущин%
Дата: 01{-}10{-}2013%
Название: None%
Подсказка: \textbackslash{}\textbackslash{}%
nan%
\fi

%
\section{}%
\label{sec:}%
Сколькими способами можно 4 белых шара, 4 синих шара и 4 красных шара
разложить в 6 различных ящиков?%
%
\iffalse%
Автор: Дмитрий Трущин%
Дата: 01{-}10{-}2013%
Название: None%
Подсказка: \textbackslash{}\textbackslash{}%
nan%
\fi

%
\section{}%
\label{sec:}%
Сколькими способами можно выложить в ряд 5 белых, 5 красных и 5 синих
шаров так, чтобы никакие два синих шара не лежали рядом?%
%
\iffalse%
Автор: Дмитрий Трущин%
Дата: 01{-}10{-}2013%
Название: None%
Подсказка: \textbackslash{}\textbackslash{}%
nan%
\fi

%
\section{}%
\label{sec:}%
В кошельке лежит а) по 20 монет; б) по 10 монет достоинством 1, 2 и 5 рублей.
Сколькими способами можно из этих монет выбрать 20?%
%
\iffalse%
Автор: Дмитрий Трущин%
Дата: 01{-}10{-}2013%
Название: None%
Подсказка: \textbackslash{}\textbackslash{}%
nan%
\fi

%
\section{}%
\label{sec:}%
Сколько решений в целых числах имеет уравнение x + y + z = 20 при условии,
что $0 \leq x, y, z \leq 10$?%
%
\iffalse%
Автор: Дмитрий Трущин%
Дата: 01{-}10{-}2013%
Название: None%
Подсказка: \textbackslash{}\textbackslash{}%
nan%
\fi

%
\section{}%
\label{sec:}%
Сколько решений в натуральных числах имеет уравнение $x y z=10^{9}$?%
%
\iffalse%
Автор: Дмитрий Трущин%
Дата: 01{-}10{-}2013%
Название: None%
Подсказка: \textbackslash{}\textbackslash{}%
nan%
\fi

%
\section{}%
\label{sec:}%
На полке стоит 12 книг. Сколькими способами можно выбрать из них 5 книг,
никакие две из которых не стоят рядом?%
%
\iffalse%
Автор: Дмитрий Трущин%
Дата: 01{-}10{-}2013%
Название: None%
Подсказка: \textbackslash{}\textbackslash{}%
nan%
\fi

%
\section{}%
\label{sec:}%
Сколькими способами можно переставить n подносов, стоящих на длинном узком столе? Это называется число перестановок из n элементов и обозначается $P_n$.%
%
\iffalse%
Автор: Ираклий Гагуа%
Дата: 07{-}01{-}2011%
Название: None%
Подсказка: \textbackslash{}\textbackslash{}%
nan%
\fi

%
\section{}%
\label{sec:}%
Сколькими способами можно поставить на шахматную доску а) белую и
черную ладьи; б) две белых ладьи так, чтобы они не били друг друга?%
%
\iffalse%
Автор: 25 школа%
Дата: 01{-}11{-}2012%
Название: None%
Подсказка: \textbackslash{}\textbackslash{}%
nan%
\fi

%
\end{document}