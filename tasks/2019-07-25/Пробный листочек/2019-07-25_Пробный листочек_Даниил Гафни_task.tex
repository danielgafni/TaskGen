\documentclass{article}%
\usepackage[T1]{fontenc}%
\usepackage[utf8]{inputenc}%
\usepackage{lmodern}%
\usepackage{textcomp}%
\usepackage{lastpage}%
\usepackage[tmargin=1cm,lmargin=4cm]{geometry}%
\usepackage[english,russian]{babel}%
\usepackage[dvipsnames]{xcolor}%
%
\title{Пробный листочек}%
\author{Даниил Гафни}%
\date{2019{-}07{-}25}%
%
\begin{document}%
\normalsize%
\maketitle%
\section{}%
\label{sec:}%
Чему равна длина окружности радиусом 1?%
%
\iffalse%
Автор: Даниил Гафни%
Дата: 2019{-}24{-}07%
Название: None%
Подсказка: \textbackslash{}\textbackslash{}%
Длину окружности можно вычислить по формуле $L=2 \pi R$%
\fi

%
\section{}%
\label{sec:}%
Найдите периментр квадрата со стороной 3%
%
\iffalse%
Автор: Даниил Гафни%
Дата: 2019{-}25{-}07%
Название: None%
Подсказка: \textbackslash{}\textbackslash{}%
Сложите длины сторон квадрата: $a + a + a + a$%
\fi

%
\section{}%
\label{sec:}%
Из шахматной доски вырезали две противоположные угловые клетки.\\
Получится ли разрезать ее на доминошки 1x2?%
%
\iffalse%
Автор: Даниил Гафни%
Дата: 2019{-}25{-}07%
Название: None%
Подсказка: \textbackslash{}\textbackslash{}%
Подсчитайте количество белых и черных клеток.%
\fi

%
\end{document}