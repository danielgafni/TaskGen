\documentclass[a4paper,11pt]{article}%
\usepackage[T1]{fontenc}%
\usepackage[utf8]{inputenc}%
\usepackage{lmodern}%
\usepackage{textcomp}%
\usepackage{lastpage}%
\usepackage[english,russian]{babel}%
\usepackage{mathtext}%
\usepackage[a4paper,margin=1.5truecm,top=1.3truecm,bottom=1.0truecm]{geometry}%
\usepackage[T2A]{fontenc}%
\usepackage{amsmath}%
\usepackage{amsthm}%
\usepackage{amsfonts}%
\usepackage{mathabx}%
\usepackage{graphicx}%
\usepackage{tabularx}%
%
\title{Случайнай комбинаторика2 {-} ответы}%
\author{Даниил Гафни}%
\date{2019{-}07{-}31}%
%
\begin{document}%
\normalsize%
\maketitle%
\section{(сложность {-} 70)}%
\label{sec:( {-} 70)}%
\hspace{3ex} Ответ: nan \\%
%
\hspace*{3ex} Решение: \\%
nan

%
\section{(сложность {-} 71)}%
\label{sec:( {-} 71)}%
\hspace{3ex} Ответ: nan \\%
%
\hspace*{3ex} Решение: \\%
nan

%
\section{(сложность {-} 72)}%
\label{sec:( {-} 72)}%
\hspace{3ex} Ответ: nan \\%
%
\hspace*{3ex} Решение: \\%
nan

%
\section{(сложность {-} 72)}%
\label{sec:( {-} 72)}%
\hspace{3ex} Ответ: nan \\%
%
\hspace*{3ex} Решение: \\%
nan

%
\section{(сложность {-} 73)}%
\label{sec:( {-} 73)}%
\hspace{3ex} Ответ: nan \\%
%
\hspace*{3ex} Решение: \\%
nan

%
\section{(сложность {-} 73)}%
\label{sec:( {-} 73)}%
\hspace{3ex} Ответ: nan \\%
%
\hspace*{3ex} Решение: \\%
nan

%
\section{(сложность {-} 73)}%
\label{sec:( {-} 73)}%
\hspace{3ex} Ответ: nan \\%
%
\hspace*{3ex} Решение: \\%
nan

%
\section{(сложность {-} 80)}%
\label{sec:( {-} 80)}%
\hspace{3ex} Ответ: nan \\%
%
\hspace*{3ex} Решение: \\%
На первое место можно положить одну из 6 карточек. Для этого есть
6 способов. В каждом из этих 6 способов на второе место можно положить одну
из оставшихся 5 карточек. Таким образом, существует $5 \cdot 6=30$ способов, чтобы
положить карточки на первое и второе места. В каждом из этих 30 способов на
третье место можно положить одну из оставшихся 4 карточек. Следовательно,
существует $4 \cdot 5 \cdot 6=120$ способов, чтобы положить карточки на первое, второе
и третье места. И так далее, пока не останется одна карточка. Таким образом,
при выкладывании карточек можно получить 6!=720 шестизначных чисел. Иногда нас может интересовать количество способов расположить не все n
элементов, а только несколько из них. Тогда цепочка из предыдущего
рассуждения оборвется на k-том шаге, а не дойдет до единицы. $n \cdot(n-1) \cdot(n-2) \cdot \ldots \cdot(n-k+1)$. Это число размещений k элементов из n можно более коротко записать: $\frac{n !}{(n-k) !}$. (Проверьте, что дробь сокращается до нужного произведения!)

%
\section{(сложность {-} 81)}%
\label{sec:( {-} 81)}%
\hspace{3ex} Ответ: nan \\%
%
\hspace*{3ex} Решение: \\%
nan

%
\section{(сложность {-} 81)}%
\label{sec:( {-} 81)}%
\hspace{3ex} Ответ: nan \\%
%
\hspace*{3ex} Решение: \\%
nan

%
\end{document}