\documentclass[a4paper,11pt]{article}%
\usepackage[T1]{fontenc}%
\usepackage[utf8]{inputenc}%
\usepackage{lmodern}%
\usepackage{textcomp}%
\usepackage{lastpage}%
\usepackage[english,russian]{babel}%
\usepackage{mathtext}%
\usepackage[a4paper,margin=1.5truecm,top=1.3truecm,bottom=1.0truecm]{geometry}%
\usepackage[T2A]{fontenc}%
\usepackage{amsmath}%
\usepackage{amsthm}%
\usepackage{amsfonts}%
\usepackage{mathabx}%
\usepackage{graphicx}%
\usepackage{tabularx}%
%
\theoremstyle{definition}%
\newtheorem{Ex}{\hspace{-25pt}\fbox{\phantom{123}} Задача}%
\title{Случайнай комбинаторика2}%
\author{Даниил Гафни}%
\date{2019{-}07{-}31}%
%
\begin{document}%
\normalsize%
\maketitle%
\begin{Ex}%
Сколькими способами можно переставить 7 подносов, стоящих на длинном узком столе?%
%
\end{Ex}%
%
\iffalse%
Автор: Ираклий Гагуа%
Дата: 07.01.2011%
Название: None%
Подсказка: \textbackslash{}\textbackslash{}%
nan%
\fi%
%
\begin{Ex}%
Докажите, что количество способов выбрать из n элементов  группу из k равно n!/(k!(n-k)!)%
%
\end{Ex}%
%
\iffalse%
Автор: 25 школа%
Дата: 01.07.2017%
Название: None%
Подсказка: \textbackslash{}\textbackslash{}%
nan%
\fi%
%
\begin{Ex}%
Назовем число симпатичным, если в его записи встречаются только
нечетные цифры. Сколько существует а) пятизначных симпатичных чисел;
б) пятизначниых несимпатичных чисел?%
%
\end{Ex}%
%
\iffalse%
Автор: 25 школа%
Дата: 01.11.2012%
Название: None%
Подсказка: \textbackslash{}\textbackslash{}%
nan%
\fi%
%
\begin{Ex}%
Сколькими способами можно выложить в ряд 5 белых, 5 красных и 5 синих
шаров так, чтобы никакие два синих шара не лежали рядом?%
%
\end{Ex}%
%
\iffalse%
Автор: Дмитрий Трущин%
Дата: 01.10.2013%
Название: None%
Подсказка: \textbackslash{}\textbackslash{}%
nan%
\fi%
%
\begin{Ex}%
Сколькими способами можно поставить на шахматную доску а) белого и
черного королей; б) двух белых королей так, чтобы они не били друг
друга?%
%
\end{Ex}%
%
\iffalse%
Автор: 25 школа%
Дата: 01.11.2012%
Название: None%
Подсказка: \textbackslash{}\textbackslash{}%
nan%
\fi%
%
\begin{Ex}%
Сколько существует десятизначных чисел, у которых сумма цифр равна 4?%
%
\end{Ex}%
%
\iffalse%
Автор: 25 школа%
Дата: 01.07.2017%
Название: None%
Подсказка: \textbackslash{}\textbackslash{}%
nan%
\fi%
%
\begin{Ex}%
Сколько решений в целых числах имеет уравнение x + y + z = 20 при условии,
что $0 \leq x, y, z \leq 10$?%
%
\end{Ex}%
%
\iffalse%
Автор: Дмитрий Трущин%
Дата: 01.10.2013%
Название: None%
Подсказка: \textbackslash{}\textbackslash{}%
nan%
\fi%
%
\begin{Ex}%
6 карточек пронумерованы числами 1, 2, 3, 4, 5, 6.
Карточки наугад выкладываем в ряд. Сколько при этом можно получить
различных шестизначных чисел?%
%
\end{Ex}%
%
\iffalse%
Автор: Дарья Анзон%
Дата: 01.07.2017%
Название: None%
Подсказка: \textbackslash{}\textbackslash{}%
nan%
\fi%
%
\begin{Ex}%
Посмотрите на строчки треугольника Паскаля.
(a) Выпишите явно, чему равно $(x + 1)^4$ и $(x + 1)^5$.
(b) Докажите, используя только реккурентное определение треугольника Паскаля, связь между биномом Ньютона и треугольником Паскаля.%
%
\end{Ex}%
%
\iffalse%
Автор: Павел Минаев%
Дата: 07.10.2017%
Название: None%
Подсказка: \textbackslash{}\textbackslash{}%
nan%
\fi%
%
\begin{Ex}%
Упростите выражения: а) $\frac{10 !}{n \cdot(n+1)} \cdot  \frac{(n+1) !}{8 !}$ б)$n ! \cdot\left(\frac{1}{n !}-\frac{1}{(n+1) !}\right)$%
%
\end{Ex}%
%
\iffalse%
Автор: Дарья Анзон%
Дата: 01.07.2017%
Название: None%
Подсказка: \textbackslash{}\textbackslash{}%
nan%
\fi%
%
\end{document}