\documentclass[a4paper,10pt]{article}%
\usepackage[T1]{fontenc}%
\usepackage[utf8]{inputenc}%
\usepackage{lmodern}%
\usepackage{textcomp}%
\usepackage{lastpage}%
\usepackage[english,russian]{babel}%
\usepackage[a4paper,margin=1.5truecm,top=1.3truecm,bottom=1.0truecm]{geometry}%
%
\title{Линейное возрастание сложности}%
\author{Даниил Гафни}%
\date{2019{-}07{-}31}%
%
\begin{document}%
\normalsize%
\maketitle%
\section{}%
\label{sec:}%
Сколько существует
А) трехзначных чисел?
Б) пятизначных чисел?
В) пятизначных чисел, в которых все цифры четные?
Г) пятизначных чисел, в которых есть хотя бы одна четная цифра?%
%
\iffalse%
Автор: Ираклий Гагуа%
Дата: 07{-}01{-}2011%
Название: None%
Подсказка: \textbackslash{}\textbackslash{}%
nan%
\fi

%
\section{}%
\label{sec:}%
Сколькими способами можно из 15 слов выбрать набор, состоящий не более чем
из 5 слов?%
%
\iffalse%
Автор: 25 школа%
Дата: 01{-}07{-}2017%
Название: None%
Подсказка: \textbackslash{}\textbackslash{}%
nan%
\fi

%
\section{}%
\label{sec:}%
В группе Платоны учатся 2 девочки и 8 мальчиков. Сколькими способами
Дмитрий Владимирович может отправить на олимпиаду команду из четырех человек
так, чтобы среди участников была хотя бы одна девочка?%
%
\iffalse%
Автор: 25 школа%
Дата: 01{-}07{-}2017%
Название: None%
Подсказка: \textbackslash{}\textbackslash{}%
nan%
\fi

%
\section{}%
\label{sec:}%
В группе Платоны учатся 2 девочки и 8 мальчиков. Сколькими способами
Дмитрий Владимирович может отправить на олимпиаду команду из четырех человек
так, чтобы среди участников была хотя бы одна девочка?%
%
\iffalse%
Автор: 25 школа%
Дата: 01{-}07{-}2017%
Название: None%
Подсказка: \textbackslash{}\textbackslash{}%
nan%
\fi

%
\section{}%
\label{sec:}%
Сколькими способами можно 10 человек разбить на 2 баскетбольные команды
по 5 в каждой?%
%
\iffalse%
Автор: 25 школа%
Дата: 01{-}07{-}2017%
Название: None%
Подсказка: \textbackslash{}\textbackslash{}%
nan%
\fi

%
\section{}%
\label{sec:}%
Сколько раз можно разделить число 100! на 10 нацело?%
%
\iffalse%
Автор: Дарья Анзон%
Дата: 01{-}07{-}2017%
Название: None%
Подсказка: \textbackslash{}\textbackslash{}%
nan%
\fi

%
\section{}%
\label{sec:}%
Посмотрите на строчки треугольника Паскаля.
(a) Выпишите явно, чему равно $(x + 1)^4$ и $(x + 1)^5$.
(b) Докажите, используя только реккурентное определение треугольника Паскаля, связь между биномом Ньютона и треугольником Паскаля.%
%
\iffalse%
Автор: Павел Минаев%
Дата: 07{-}10{-}2017%
Название: None%
Подсказка: \textbackslash{}\textbackslash{}%
nan%
\fi

%
\section{}%
\label{sec:}%
В этой задаче нужно внимательно смотреть на треугольник Паскаля
(a) Пусть есть бесконечная шахматная доска вправо и вверх. В левом нижнем углу стоит хромой король,
умеющий ходить только вправо и вверх. Сколько есть способов добраться до клетки (n;m), где n – число
строк по вертикали, а m – по горизонтали?
(b) Напишем в каждую клетку доски из предыдущей задачи число способов добраться до этой клетки. Как
связан треугольник Паскаля и эта табличка с числами? Докажите свою гипотезу.%
%
\iffalse%
Автор: Павел Минаев%
Дата: 07{-}10{-}2017%
Название: None%
Подсказка: \textbackslash{}\textbackslash{}%
nan%
\fi

%
\section{}%
\label{sec:}%
В этой задаче нужно внимательно смотреть на треугольник Паскаля
(a) Пусть есть бесконечная шахматная доска вправо и вверх. В левом нижнем углу стоит хромой король,
умеющий ходить только вправо и вверх. Сколько есть способов добраться до клетки (n;m), где n – число
строк по вертикали, а m – по горизонтали?
(b) Напишем в каждую клетку доски из предыдущей задачи число способов добраться до этой клетки. Как
связан треугольник Паскаля и эта табличка с числами? Докажите свою гипотезу.%
%
\iffalse%
Автор: Павел Минаев%
Дата: 07{-}10{-}2017%
Название: None%
Подсказка: \textbackslash{}\textbackslash{}%
nan%
\fi

%
\section{}%
\label{sec:}%
Раскроем скобки в выражении $(x + 1)^n$
(a) Пусть все слагаемые приведены. Найдите коэффициент при $x^m$ с помощью комбинаторных размышлений.
(b) Докажите, что этот коэффициент равен найденному в предыдущем пункте числу, по индукции.
(c) Сколько будет слагаемых без приведения подобных слагаемых?
(d) Расскройте скобки у выражения $(x + y)^n$ и выразите коэффициенты прикаждом приведенном члене.
Выражение, полученное Вами в пункте (d) называется биномом Ньютона.%
%
\iffalse%
Автор: Павел Минаев%
Дата: 07{-}10{-}2017%
Название: None%
Подсказка: \textbackslash{}\textbackslash{}%
nan%
\fi

%
\end{document}