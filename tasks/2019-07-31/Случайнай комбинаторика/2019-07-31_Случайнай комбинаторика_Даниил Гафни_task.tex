\documentclass[a4paper,11pt]{article}%
\usepackage[T1]{fontenc}%
\usepackage[utf8]{inputenc}%
\usepackage{lmodern}%
\usepackage{textcomp}%
\usepackage{lastpage}%
\usepackage[english,russian]{babel}%
\usepackage{mathtext}%
\usepackage[a4paper,margin=1.5truecm,top=1.3truecm,bottom=1.0truecm]{geometry}%
\usepackage[T2A]{fontenc}%
\usepackage{amsmath}%
\usepackage{amsthm}%
\usepackage{amsfonts}%
\usepackage{mathabx}%
\usepackage{graphicx}%
\usepackage{tabularx}%
%
\theoremstyle{definition}%
\newtheorem{Ex}{\hspace{-25pt}\fbox{\phantom{123}} Задача}%
\title{Случайнай комбинаторика}%
\author{Даниил Гафни}%
\date{2019{-}07{-}31}%
%
\begin{document}%
\normalsize%
\maketitle%
\begin{Ex}%
Рота состоит из 3 лейтенантов, 6 сержантов и 60 рядовых. Сколькими способами
можно выбрать из них отряд, состоящий из одного лейтенанта, двух сержантов и 20
рядовых?%
%
\end{Ex}%
%
\iffalse%
Автор: 25 школа%
Дата: 01.07.2017%
Название: None%
Подсказка: \textbackslash{}\textbackslash{}%
nan%
\fi%
%
\begin{Ex}%
Докажите, что количество способов выбрать из n элементов  группу из k равно n!/(k!(n-k)!)%
%
\end{Ex}%
%
\iffalse%
Автор: 25 школа%
Дата: 01.07.2017%
Название: None%
Подсказка: \textbackslash{}\textbackslash{}%
nan%
\fi%
%
\begin{Ex}%
На прямой отмечено 10 точек, а на параллельной ей прямой –11. Сколько
существует трапеций с вершинами в этих точках?%
%
\end{Ex}%
%
\iffalse%
Автор: 25 школа%
Дата: 01.07.2017%
Название: None%
Подсказка: \textbackslash{}\textbackslash{}%
nan%
\fi%
%
\begin{Ex}%
Имеется 6 разноцветных коробок. Сколькими способами можно в эти коробки
разложить 20 шаров при условии, что:
а. Шары разных размеров.
б. Шары одинаковые.
в. Шары одинаковые, и не должно остаться пустых коробок.
г. Шары одинаковые, и в каждой коробке должно оказаться не менее двух шаров.%
%
\end{Ex}%
%
\iffalse%
Автор: Дмитрий Трущин%
Дата: 01.10.2013%
Название: None%
Подсказка: \textbackslash{}\textbackslash{}%
nan%
\fi%
%
\begin{Ex}%
Сколько существует двоичных слов длины n, в которых ровно k нулей? Это называется число сочетаний из n элементов по k и обозначается $C_n^k$ или $(^n_k)$%
%
\end{Ex}%
%
\iffalse%
Автор: Ираклий Гагуа%
Дата: 07.01.2011%
Название: None%
Подсказка: \textbackslash{}\textbackslash{}%
nan%
\fi%
%
\begin{Ex}%
В классе из 25 человек требуется выбрать а) старосту и его заместителя; б)
двух дежурных. Сколькими способами это можно сделать?%
%
\end{Ex}%
%
\iffalse%
Автор: 25 школа%
Дата: 01.11.2012%
Название: None%
Подсказка: \textbackslash{}\textbackslash{}%
nan%
\fi%
%
\begin{Ex}%
Сколько диагоналей имеет выпуклый а) 25-тиугольник б) n-угольник?%
%
\end{Ex}%
%
\iffalse%
Автор: 25 школа%
Дата: 01.11.2012%
Название: None%
Подсказка: \textbackslash{}\textbackslash{}%
nan%
\fi%
%
\begin{Ex}%
Каких семизначных чисел больше: тех, в записи которых есть 1 или
остальных?%
%
\end{Ex}%
%
\iffalse%
Автор: 25 школа%
Дата: 01.11.2012%
Название: None%
Подсказка: \textbackslash{}\textbackslash{}%
nan%
\fi%
%
\begin{Ex}%
Сколько решений в натуральных числах имеет уравнение $x y z=10^{9}$?%
%
\end{Ex}%
%
\iffalse%
Автор: Дмитрий Трущин%
Дата: 01.10.2013%
Название: None%
Подсказка: \textbackslash{}\textbackslash{}%
nan%
\fi%
%
\begin{Ex}%
На гранях кубика расставлены числа от 1 до 6. Кубик бросили два раза. В первый раз сумма чисел на четырёх боковых гранях оказалась равна 12, во второй — 15. Какое число написано на грани, противоположной той, где написана цифра 3? %
%
\end{Ex}%
%
\iffalse%
Автор: Дарья Анзон%
Дата: 01.07.2017%
Название: None%
Подсказка: \textbackslash{}\textbackslash{}%
nan%
\fi%
%
\end{document}