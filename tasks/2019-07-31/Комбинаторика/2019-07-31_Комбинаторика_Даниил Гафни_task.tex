\documentclass[a4paper,11pt]{article}%
\usepackage[T1]{fontenc}%
\usepackage[utf8]{inputenc}%
\usepackage{lmodern}%
\usepackage{textcomp}%
\usepackage{lastpage}%
\usepackage[english,russian]{babel}%
\usepackage{mathtext}%
\usepackage[a4paper,margin=1.5truecm,top=1.3truecm,bottom=1.0truecm]{geometry}%
\usepackage[T2A]{fontenc}%
\usepackage{amsmath}%
\usepackage{amsthm}%
\usepackage{amsfonts}%
\usepackage{mathabx}%
\usepackage{graphicx}%
\usepackage{tabularx}%
%
\theoremstyle{definition}%
\newtheorem{Ex}{\hspace{-25pt}\fbox{\phantom{123}} Задача}%
\title{Комбинаторика}%
\author{Даниил Гафни}%
\date{2019{-}07{-}31}%
%
\begin{document}%
\normalsize%
\maketitle%
\begin{Ex}%
Сколько существует
А) трехзначных чисел?
Б) пятизначных чисел?
В) пятизначных чисел, в которых все цифры четные?
Г) пятизначных чисел, в которых есть хотя бы одна четная цифра?%
%
\end{Ex}%
%
\iffalse%
Автор: Ираклий Гагуа%
Дата: 07.01.2011%
Название: None%
Подсказка: \textbackslash{}\textbackslash{}%
nan%
\fi%
%
\begin{Ex}%
Сколько существует двузначных чисел, у которых цифры идут по возрастанию?%
%
\end{Ex}%
%
\iffalse%
Автор: Ираклий Гагуа%
Дата: 07.01.2011%
Название: None%
Подсказка: \textbackslash{}\textbackslash{}%
nan%
\fi%
%
\begin{Ex}%
Сколькими способами можно выбрать гласную и согласную буквы в слове «КРУЖОК»?%
%
\end{Ex}%
%
\iffalse%
Автор: Ираклий Гагуа%
Дата: 07.01.2011%
Название: None%
Подсказка: \textbackslash{}\textbackslash{}%
nan%
\fi%
%
\begin{Ex}%
В магазин «Всё для чая» завезли 4 различные чайные ложки.
а) Сколькими способами можно купить набор из трех предметов?
б) А из двух предметов?%
%
\end{Ex}%
%
\iffalse%
Автор: Ираклий Гагуа%
Дата: 07.01.2011%
Название: None%
Подсказка: \textbackslash{}\textbackslash{}%
nan%
\fi%
%
\begin{Ex}%
Каких двоичных слов длины 50 больше: тех, в которых 36 единиц, или тех, в которых 14 единиц?%
%
\end{Ex}%
%
\iffalse%
Автор: Ираклий Гагуа%
Дата: 07.01.2011%
Название: None%
Подсказка: \textbackslash{}\textbackslash{}%
nan%
\fi%
%
\begin{Ex}%
Рота состоит из 3 лейтенантов, 6 сержантов и 60 рядовых. Сколькими способами
можно выбрать из них отряд, состоящий из одного лейтенанта, двух сержантов и 20
рядовых?%
%
\end{Ex}%
%
\iffalse%
Автор: 25 школа%
Дата: 01.07.2017%
Название: None%
Подсказка: \textbackslash{}\textbackslash{}%
nan%
\fi%
%
\begin{Ex}%
Сколькими способами можно переставить n подносов, стоящих на длинном узком столе? Это называется число перестановок из n элементов и обозначается $P_n$.%
%
\end{Ex}%
%
\iffalse%
Автор: Ираклий Гагуа%
Дата: 07.01.2011%
Название: None%
Подсказка: \textbackslash{}\textbackslash{}%
nan%
\fi%
%
\begin{Ex}%
Переплетчик должен переплести 12 одинаковых книг в красный, синий или
зеленый переплеты. Сколькими способами он может это сделать?%
%
\end{Ex}%
%
\iffalse%
Автор: Дмитрий Трущин%
Дата: 01.10.2013%
Название: None%
Подсказка: \textbackslash{}\textbackslash{}%
nan%
\fi%
%
\begin{Ex}%
Сколькими способами можно поставить на шахматную доску а) белую и
черную ладьи; б) две белых ладьи так, чтобы они не били друг друга?%
%
\end{Ex}%
%
\iffalse%
Автор: 25 школа%
Дата: 01.11.2012%
Название: None%
Подсказка: \textbackslash{}\textbackslash{}%
nan%
\fi%
%
\begin{Ex}%
Сколько решений в натуральных числах имеет уравнение $x y z=10^{9}$?%
%
\end{Ex}%
%
\iffalse%
Автор: Дмитрий Трущин%
Дата: 01.10.2013%
Название: None%
Подсказка: \textbackslash{}\textbackslash{}%
nan%
\fi%
%
\end{document}