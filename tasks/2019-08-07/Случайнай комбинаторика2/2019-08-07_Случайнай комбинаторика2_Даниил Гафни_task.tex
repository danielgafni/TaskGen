\documentclass[a4paper,11pt]{article}%
\usepackage[T1]{fontenc}%
\usepackage[utf8]{inputenc}%
\usepackage{lmodern}%
\usepackage{textcomp}%
\usepackage{lastpage}%
\usepackage[english,russian]{babel}%
\usepackage{mathtext}%
\usepackage[a4paper,margin=1.5truecm,top=1.3truecm,bottom=1.0truecm]{geometry}%
\usepackage[T2A]{fontenc}%
\usepackage{amsmath}%
\usepackage{amsthm}%
\usepackage{amsfonts}%
\usepackage{mathabx}%
\usepackage{graphicx}%
\usepackage{tabularx}%
%
\theoremstyle{definition}%
\newtheorem{Ex}{\hspace{-25pt}\fbox{\phantom{123}} Задача}%
\title{Случайнай комбинаторика2}%
\author{Даниил Гафни}%
\date{2019{-}08{-}07}%
%
\begin{document}%
\normalsize%
\maketitle%
\begin{Ex}%
В команде из 11 человек нужно выбрать капитана и заместителя. Сколькими способами это можно сделать?%
%
\end{Ex}%
%
\iffalse%
Автор: Ираклий Гагуа\newline%
%
Дата: 07.01.2011\newline%
%
Название: \newline%
%
Подсказка: \textbackslash{}\textbackslash{}%
nan%
\fi%
%
\begin{Ex}%
В магазин «Всё для чая» завезли 4 различные чайные ложки.
а) Сколькими способами можно купить набор из трех предметов?
б) А из двух предметов?%
%
\end{Ex}%
%
\iffalse%
Автор: Ираклий Гагуа\newline%
%
Дата: 07.01.2011\newline%
%
Название: \newline%
%
Подсказка: \textbackslash{}\textbackslash{}%
nan%
\fi%
%
\begin{Ex}%
Сколькими способами можно поставить на доску 3х3 двух белых ладей так, чтобы они не били друг друга?%
%
\end{Ex}%
%
\iffalse%
Автор: Ираклий Гагуа\newline%
%
Дата: 07.01.2011\newline%
%
Название: \newline%
%
Подсказка: \textbackslash{}\textbackslash{}%
nan%
\fi%
%
\begin{Ex}%
Сколькими способами можно переставить n подносов, стоящих на длинном узком столе? Это называется число перестановок из n элементов и обозначается $P_n$.%
%
\end{Ex}%
%
\iffalse%
Автор: Ираклий Гагуа\newline%
%
Дата: 07.01.2011\newline%
%
Название: \newline%
%
Подсказка: \textbackslash{}\textbackslash{}%
nan%
\fi%
%
\begin{Ex}%
Семен имеет 6 друзей и в течение 5 дней приглашает в гости каких–то троих
из них так, чтобы компания ни разу не повторялась. Сколькими способами он может
это сделать?%
%
\end{Ex}%
%
\iffalse%
Автор: 25 школа\newline%
%
Дата: 01.07.2017\newline%
%
Название: \newline%
%
Подсказка: \textbackslash{}\textbackslash{}%
nan%
\fi%
%
\begin{Ex}%
Сколько существует двоичных слов длины 10, в которых ровно 4 нуля?%
%
\end{Ex}%
%
\iffalse%
Автор: Ираклий Гагуа\newline%
%
Дата: 07.01.2011\newline%
%
Название: \newline%
%
Подсказка: \textbackslash{}\textbackslash{}%
nan%
\fi%
%
\begin{Ex}%
Сколькими способами можно поставить на шахматную доску а) белую и
черную ладьи; б) две белых ладьи так, чтобы они не били друг друга?%
%
\end{Ex}%
%
\iffalse%
Автор: 25 школа\newline%
%
Дата: 01.11.2012\newline%
%
Название: \newline%
%
Подсказка: \textbackslash{}\textbackslash{}%
nan%
\fi%
%
\begin{Ex}%
Сколько существует десятизначных чисел, у которых сумма цифр равна 4?%
%
\end{Ex}%
%
\iffalse%
Автор: 25 школа\newline%
%
Дата: 01.07.2017\newline%
%
Название: \newline%
%
Подсказка: \textbackslash{}\textbackslash{}%
nan%
\fi%
%
\begin{Ex}%
Сколько решений в целых числах имеет уравнение a + b + c + d + e + f = 20 при
условии, что
а) $a, b, c, d, e, f \geq 0$ б)$a, b, c, d, e, f \geq 1$ в) $a, b, c, d, e, f \geq 2$%
%
\end{Ex}%
%
\iffalse%
Автор: Дмитрий Трущин\newline%
%
Дата: 01.10.2013\newline%
%
Название: \newline%
%
Подсказка: \textbackslash{}\textbackslash{}%
nan%
\fi%
%
\begin{Ex}%
Сколько решений в целых числах имеет уравнение x + y + z = 20 при условии,
что $0 \leq x, y, z \leq 10$?%
%
\end{Ex}%
%
\iffalse%
Автор: Дмитрий Трущин\newline%
%
Дата: 01.10.2013\newline%
%
Название: \newline%
%
Подсказка: \textbackslash{}\textbackslash{}%
nan%
\fi%
%
\end{document}